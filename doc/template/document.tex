%%%%%%%%%%%%%%%%%%%%%%%%%%%%%%%%%%%%%%%%%%%%%%%%%%%%%%%%%%%%%%%%%%%%%%%%%%%%%%%%%%%%%%%%%%%%%%%%
% UNIVOTE LATEX TEMPLATE
%%%%%%%%%%%%%%%%%%%%%%%%%%%%%%%%%%%%%%%%%%%%%%%%%%%%%%%%%%%%%%%%%%%%%%%%%%%%%%%%%%%%%%%%%%%%%%%%

\documentclass[bibtotoc,halfparskip,oneside]{scrreprt}

\usepackage{import}
\import{../latex/}{german.tex} 
% you must set the path according to the current document folder
% use "german.tex" for documents in german and "english.tex" for documents in english
\inputpath{{listings/}{figures/}}

\begin{document}

\title{Dokument-Template}
\maketitle

\begin{versionhistory}
	\vhEntry{0.1}{10.05.2012}{Rolf Haenni}{Initiales Dokument, welches als Template für andere Dokumente dienen soll.}
	\vhEntry{0.2}{12.05.2012}{Stephan Fischli|Eric Dubuis}{Kleine Änderungen.}
\end{versionhistory}

%%%%%%%%%%%%%%%%%%%%%%%%%%%%%%%%%%%%%%%%%%%%%%%%%%%%%%%%%%%%%%%%%%%%%%%%%%%%%%%%%%%%%%%%%%%%%%%%
\chapter*{Zusammenfassung}
%%%%%%%%%%%%%%%%%%%%%%%%%%%%%%%%%%%%%%%%%%%%%%%%%%%%%%%%%%%%%%%%%%%%%%%%%%%%%%%%%%%%%%%%%%%%%%%%

\lipsum[1-3]

%%%%%%%%%%%%%%%%%%%%%%%%%%%%%%%%%%%%%%%%%%%%%%%%%%%%%%%%%%%%%%%%%%%%%%%%%%%%%%%%%%%%%%%%%%%%%%%%

\tableofcontents

%%%%%%%%%%%%%%%%%%%%%%%%%%%%%%%%%%%%%%%%%%%%%%%%%%%%%%%%%%%%%%%%%%%%%%%%%%%%%%%%%%%%%%%%%%%%%%%%
\chapter{Kapitel}
%%%%%%%%%%%%%%%%%%%%%%%%%%%%%%%%%%%%%%%%%%%%%%%%%%%%%%%%%%%%%%%%%%%%%%%%%%%%%%%%%%%%%%%%%%%%%%%%

\lipsum[1] See \cite{HS11,JCJ05,SH10} for more information.

%===============================================================================================
\section{Java Sourcecode}
%===============================================================================================

\lipsum[1-3]

\begin{lstlisting}[style=javastyle,caption={Beispiel von Java Sourcecode}]
import java.util.Scanner;
 
public class EarthquakeTester
{  
	public static void main(String[] args)
	{  
		Scanner in = new Scanner(System.in);
		System.out.print("Enter a magnitude on the Richter scale: "); 
		double magnitude = in.nextDouble();
		Earthquake quake = new Earthquake(magnitude);
		System.out.println(quake.getDescription());
	}
}
\end{lstlisting}

Hier ist ein Beispiel, das zeigt, wie Java-Code in Text eingebunden wird, also zum Beispiel so: \lstinline[style=javastyle]$class EarthquakeTester$.

\lstinputlisting[style=javastyle,firstline=3,lastline=10,caption={Beispiel von Java Sourcecode aus einer externen Datei}]{EarthquakeTester.java}

\lipsum[1]


%===============================================================================================
\section{Bilder}
%===============================================================================================

\lipsum[1-3]

\includefigure{diagram}{scale=1}{label-fig1}{Beispiel eines Sequenz-Diagrams (.pdf, .png, oder .jpg).}
% Arguments: filename, options, label, caption

%===============================================================================================
\section{Aufzählungen}
%===============================================================================================

\lipsum[1]

%-----------------------------------------------------------------------------------------------
\subsection{Ohne Labels}
%-----------------------------------------------------------------------------------------------

\lipsum[1]

\begin{itemize}
	\item \lipsum[3]
	\item \lipsum[3]
\end{itemize}

\lipsum[1]

\begin{enumerate}
	\item \lipsum[3]
	\item \lipsum[3]
\end{enumerate}

%-----------------------------------------------------------------------------------------------
\subsection{Mit Labels}
%-----------------------------------------------------------------------------------------------

\lipsum[1]

\begin{description}
	\item[Variante 1.] \lipsum[3]
	\item[Variante 2.] \lipsum[3]
\end{description}

\lipsum[1]

\begin{labeling}{Variante Lang:}
	\item[Variante 1:] \lipsum[3]
	\item[Variante 2:] \lipsum[3]
	\item[Variante Lang:] \lipsum[3]
\end{labeling}

%-----------------------------------------------------------------------------------------------
\subsection{Mini-Sections}
%-----------------------------------------------------------------------------------------------

\lipsum[1]

\minisec{Variante 1}
\lipsum[1]

\minisec{Variante 2}
\lipsum[1]


%%%%%%%%%%%%%%%%%%%%%%%%%%%%%%%%%%%%%%%%%%%%%%%%%%%%%%%%%%%%%%%%%%%%%%%%%%%%%%%%%%%%%%%%%%%%%%%%
\chapter{Kapitel}
%%%%%%%%%%%%%%%%%%%%%%%%%%%%%%%%%%%%%%%%%%%%%%%%%%%%%%%%%%%%%%%%%%%%%%%%%%%%%%%%%%%%%%%%%%%%%%%%

\lipsum[1]

%===============================================================================================
\section{Abschnitt}
%===============================================================================================

\lipsum[1-3]

%===============================================================================================
\section{Abschnitt}
%===============================================================================================

\lipsum[1-3]

%%%%%%%%%%%%%%%%%%%%%%%%%%%%%%%%%%%%%%%%%%%%%%%%%%%%%%%%%%%%%%%%%%%%%%%%%%%%%%%%%%%%%%%%%%%%%%%%
\appendix\chapter{Ein Anhang}
%%%%%%%%%%%%%%%%%%%%%%%%%%%%%%%%%%%%%%%%%%%%%%%%%%%%%%%%%%%%%%%%%%%%%%%%%%%%%%%%%%%%%%%%%%%%%%%%

\lipsum

%%%%%%%%%%%%%%%%%%%%%%%%%%%%%%%%%%%%%%%%%%%%%%%%%%%%%%%%%%%%%%%%%%%%%%%%%%%%%%%%%%%%%%%%%%%%%%%%
\bibliographystyle{plain} \bibliography{bibtex/library}
%%%%%%%%%%%%%%%%%%%%%%%%%%%%%%%%%%%%%%%%%%%%%%%%%%%%%%%%%%%%%%%%%%%%%%%%%%%%%%%%%%%%%%%%%%%%%%%%

\end{document}
